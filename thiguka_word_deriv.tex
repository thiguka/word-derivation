\documentclass{article}
\usepackage{multicol}
\usepackage[margin=4em]{geometry}
\usepackage{gb4e}
\usepackage{times}
\usepackage{array}
\usepackage{textcomp}

\newcommand{\textapprox}{\raisebox{0.5ex}{\texttildelow}}
\newcommand{\agradj}{AGR\textapprox{}ADJ}

\title{Word Derivation in Thiguka}
\author{Lemuria}

\begin{document}
\maketitle

\begin{multicols}{2}
\section{Background}
Thiguka is a constructed language created by Lemuria in 2024.
This paper describes common Thiguka methods of deriving new words from already-existing roots.

Thiguka uses multiple methods to derive new words:

\begin{enumerate}
    \item Loanwords from other languages, such as \emph{balu} (`light blue'), derived from English `blue'.
    \item Shortening of multi-adjective phrases.
    \item Lemuria making up new roots on the spot.
\end{enumerate}

\subsection{Glossing}
This paper uses AGR to mean `agreement' when glossing Thiguka adjectives.


\subsection{Adjectives}
Thiguka uses reduplication for adjectives, attaching the first syllable of the modified word followed by \emph{-gu-} to the modifier.

\begin{exe}
    \ex \gll tu\textapprox{}gu-elo tubi\\
             \agradj{}-yellow water\\
        \glt `urine'
\end{exe}

There are other ways to create adjectives, with multiple prefixes and suffixes available for use.\\

Thiguka has the agent prefix \emph{fah-}, which is equivalent to English \emph{-er}.\\
\begin{exe}
    \ex \gll fah-lahkela\\
    AGT-make\\
    \glt `maker', `creator'
\end{exe}

\section{Examples}
\subsection{Phrases}
The shortening of long strings of adjectives is a common word derivation method in Thiguka, with long word compounds being shortened.
Usually, a syllable is taken from the front of each modifying word and then attached to the final few syllables of the modified word.

\begin{exe}
    \ex{} \gll thi\textapprox{}gu-asalisa alu thi-gu-luti thil\\
                 AGR\textapprox{}ADJ-sun and AGR-ADJ-moon thing\\
                 `asalilutil'
    \glt{}       `solar eclipse'
\end{exe}

Often, longer phrases are condensed.

\begin{exe}
    \ex{} \gll pelu-pah-sa kufa-sa pelu-tay-elah kolaga-sa tha\textapprox{}gu-ithere tharasag-tay-elah\\
                 person-NOM-SG compel-PRS person-ACC-PL comply-PRS \agradj{}-internet rule-ACC-PL\\
                 `pekulugusag'
    \glt{}       `internet moderator'
    \glt{}       (lit. `person who makes people comply with internet rules')
\end{exe}

Thiguka also has no explicit rule against shortening and combining already-compounded noun phrases. These words are treated like roots in Thiguka and can be derived like any other.

\begin{exe}
    \ex{} \gll  pe\textapprox{}gu-palad pekulugusag\\
                \agradj{}-bad internet.moderator\\
    \glt{}      `bad moderator', `annoying moderator'
\end{exe}

\begin{exe}
    \ex{} \gll  fah-lahkela-pah pekulugusag-tay-elah\\
                AGT-make-NOM internet.moderator-ACC-PL\\
    \glt{}      `internet moderator trainer'
    \glt{}      lit. `maker of internet moderators'
\end{exe}

For denoting administrators, who are generally ranked higher than moderators in most internet communities:

\begin{exe}
    \ex{} \gll  ri-pekulugusag\\
                INT-internet.moderator\\
    \glt{}      `admin'
    \glt{}      lit. `super internet moderator'
\end{exe}

\subsection{Loanwords}
Often, Thiguka will borrow either a word from English or Tagalog, the natlangs in which Lemuria is most fluent.
In this case, the new word entering the language is modified for compliance with Thiguka phonotactics.

\begin{center}
\begin{tabular}{ m{1cm} m{1cm} | m{4cm} }
    Thiguka & English & Etymology\\
    \hline
    balu & blue  & English \emph{blue}\\
    pasa & cat   & Tagalog \emph{pusa} (``cat'')\\
    pudi & white & Tagalog\,\emph{puti}\,(``white'')\\
\end{tabular}
\end{center}

\end{multicols}
\end{document}